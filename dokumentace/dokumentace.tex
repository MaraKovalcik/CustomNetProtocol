\documentclass[11pt]{article}
\fontfamily{Times}

\usepackage[left=2cm,right=2cm,text={17cm, 24cm},top=2.cm]{geometry}
\usepackage[czech]{babel}
\usepackage[utf8x]{inputenc}
\usepackage{scrextend}
\usepackage{lipsum}
\usepackage{graphicx}
\usepackage{amsthm}
\usepackage{color}

%\usepackage{titlesec}
%\titlespacing\section{0pt}{12pt plus 4pt minus 2pt}{10pt plus 2pt minus 2pt}
%\titlespacing\subsection{20pt}{12pt plus 4pt minus 2pt}{10pt plus 2pt minus 2pt}
%\titlespacing\subsubsection{40pt}{12pt plus 4pt minus 2pt}{10pt plus 2pt minus 2pt}

\title{Počítačové komunikace a sítě\\[0.5 cm] Dokumentace k projektu 1}		
\author{Marek Kovalčík}								
\date{8. březen 2017}								

\makeatletter
\let\thetitle\@title
\let\theauthor\@author
\let\thedate\@date
\makeatother

\usepackage{tocloft}
\usepackage{changepage}


\renewcommand\cfttoctitlefont{\hfill\Large\bfseries}
\renewcommand\cftaftertoctitle{\hfill\mbox{}}


\setcounter{tocdepth}{2}

\begin{document}
	
	%%%%%%%%%%%%%%%%%%%%%%%%%%%%%%%%%%%%%%%%%%%%%%%%%%%%%%%%%%%%%%%%%%%%%
	
	\begin{titlepage}
		\centering
		\vspace*{0.5 cm}
		\includegraphics[scale = 0.3]{logo.png}\\[2.0 cm]
		\textsc{\LARGE Vysoké učení technické v Brně}\\[0.3 cm]
		\textsc{\Large Fakulta informačních technologií}\\[2.0 cm]
		\rule{\linewidth}{0.2 mm} \\[2.0 cm]
		{ \huge \bfseries \thetitle}\\[2.0 cm]
		\LARGE{Varianta 1: Klient-server pro získání informace o uživatelích}\\[0.5 cm]
		\rule{\linewidth}{0.2 mm} \\[2.0 cm]
		
		\begin{minipage}{0.4\textwidth}
			\begin{flushleft} \large
				\vspace{1 cm}
				Marek Kovalčík
			\end{flushleft}
		\end{minipage}
		\begin{minipage}{0.4\textwidth}
			\begin{flushright} \large
				%\emph{Login a ID:} 
				\vspace{1 cm}
				xkoval14, 196248 \hspace{1 cm}\\								
			\end{flushright}
		\end{minipage}\\[2 cm]
		
		{\large \thedate}\\[2 cm]
		
		\vfill
		
	\end{titlepage}
	%%%%%%%%%%%%%%%%%%%%%%%%%%%%%%%%%%%%%%%%%%%%%%%%%%%%%%%%%%%%%%
	\newpage
	\tableofcontents
	\clearpage
	
	
	%%%%%%%%%%%%%%%%%%%%%%%%%%%%%%%%%%%%%%%%%%%%%%%%%%%%%%%%%%%%%%
	
	\section{Zadání}
	\begin{flushleft}
		Zadáním tohoto projektu bylo vytvořit aplikaci v jazyce C/C++ pro klienta a server s použitím BSD socketů pro přenos informací o uživateli/uživatelích ze serveru. \par		
		
		\begin{center}
			Spuštění klienta: ./ipk-client -h host -p port [-n$\mid$-f$\mid$-l] login\\[0.5 cm]
			\begin{itemize}
				\item host (IP adresa nebo fully-qualified DNS name) identifikace serveru jakožto koncového bodu komunikace klienta;
				port (číslo) cílové číslo portu;
				\item -n značí, že bude vráceno plné jméno uživatele včetně případných dalších informací pro uvedený login (User ID Info);
				\item -f značí, že bude vrácena informace o domácím adresáři uživatele pro uvedený login (Home directory);
				\item -l značí, že bude vrácen seznam všech uživatelů, tento bude vypsán tak, že každé uživatelské jméno bude na zvláštním řádku; v tomto případě je login nepovinný. Je-li však uveden bude použit jako prefix pro výběr uživatelů.
				\item login určuje přihlašovací jméno uživatele pro výše uvedené operace.
			\end{itemize}
		\end{center}
		\begin{center}
			Spuštění serveru: ./ipk-server -p port  \\[0.5 cm]
			\begin{itemize}
				\item port (číslo) číslo portu, na kterém server naslouchá na připojení od klientů.
			\end{itemize}
		\end{center}
			
	\end{flushleft}
	
	\section{Trocha teorie}
	\begin{flushleft}
		Síťový socket (anglicky network socket) je v informatice koncový bod připojení přes počítačovou síť. S rozvojem Internetu většina komunikace mezi počítači používá rodinu protokolů TCP/IP. Vlastní přenos zajišťuje IP protokol, a proto je používáno i označení internetový socket. Vlastní socket je handle (abstraktní odkaz), který může program použít při volání síťového rozhraní pro programování aplikací (API), například ve funkci „odeslat tato data na tento socket“. Sockety jsou vnitřně často jen celá čísla, která odkazují do tabulky aktivních spojení.\\[0.5 cm]
		
		Počítačové procesy, poskytující aplikační služby, jsou označovány jako servery a při startu vytvářejí sockety, které jsou ve stavu naslouchání (listen). Tyto sockety čekají na iniciativu od klientských programů.\\[0.5 cm]
		
		TCP server může obsluhovat více klientů současně pomocí podřízeného procesu vytvořeného pro každého klienta a navázání TCP spojení mezi podřízeným procesem a klientem. Pro každé spojení jsou vytvořeny unikátní specializované sockety. Když je zaváděno socket-to-socket virtuální spojení nebo virtuální okruh (VC), také známo jako TCP relace, pomocí dálkového socketu, což poskytuje duplexní proud bytů, jsou tyto unikátní specializované sockety ve stavu spojeno (connected).[3]\\[0.5 cm] 			
	\end{flushleft}
	
	\newpage
	\section{Popis řešení aplikačního protokolu}
		\subsection{Implemetace klientské aplikace}
			\begin{flushleft}
				Pro spuštění klienta je nutné zadat správně argumenty, v opačném případě se klient nespustí.	Před začátkem komunikace musí být server již spuštěný. Komunikace vždy začíná klient odesláním welcome socketu na server.\\[0.5 cm] 
				Jakmile je spojení navázáno, odešle klient zprávu, ve které první písmeno označuje požadovanou operaci a zbytek zprávy označuje hledaný login či prefix. V případě přepínače -l se loginy přijímají po jednom s tím, že klient vždy pošle zprávu zpět serveru o úspěšném přijetí aby server mohl pokračovat v posílání. Jakmile server odešle poslední login a příjde mu potvrzující zpráva, odešle zprávu end-communication , klient ji obdrží a komunikace se uzavře.\\[0.5 cm] 
				Pokouší-li se klient připojit na nějaký neočekávaný server, např. -h merlin.fit.vutbr.cz -p 80 (webový server) a nedostává žádnou odpověď, nebude čekat nekonečně dlouho, ale ukončí se po 5 vteřinách.\par		
			\end{flushleft}
			
		\subsection{Implementace serverové aplikace}
			\begin{flushleft}
				Serverová aplikace musí být spuštěná před tím, než se na ni klient/i začnou připojovat. Pro spuštění je třeba zadat přepínač -p a číslo portu. Budou-li argumenty zadány nesprávně, server se nespustí.\\[0.5 cm]
				Po spuštění server čeká v nekonečné smyčce na požadavky klientů a v jeden moment je schopen obsluhovat více klientů, tzn., že je konkurentní. Konkurence schopnost je zabezpečená pomocí procesů.\\[0.5 cm]
				Server obdrží od klienta takovou zprávu, že první písmeno určuje funkci, která se má provést a zbytek zprávy určuje hledaný login nebo prefix. $($např. nxkoval14 nebo frysavy nebo l $($v případě žádného prefixu$))$\\[0.5 cm]
				V případě nenalezení shody v souboru /etc/passwd na serveru se vrátí informační zpráva o nenalezení.\\[0.5 cm]
				V případě, že klient žádá o list uživatelů s určitým prefixem, každý login je posílán samostatně a server zpátky od klienta obdrží zprávu o úspěšném přijetí aby mohl pokračovat.\par
							
			\end{flushleft}	

	
	%%%%%%%%%%%%%%%%%%%%%%%%%%%%%%%%%%%%%%%%%%%%%%%%%%%%%%%%%%%%%%
		
	\section{Závěr}
		
			\begin{adjustwidth}{1cm}{0cm}
				\subsection{Zhodnocení}
				\begin{flushleft}
				Tento aplikační protokol je funkční, splňuje body v zadání a odchytává spoustu neočekávaných stavů a vypořádá se s nimi. Celkově jsem se svým řešením spokojen.\par
				\end{flushleft}
			\end{adjustwidth}						
	\vfill
	
	%%%%%%%%%%%%%%%%%%%%%%%%%%%%%%%%%%%%%%%%%%%%%%%%%%%%%%%%%%%%%%
	\newpage		
	\section{Přílohy}
			Klienta i server jsem testoval v mnoha varintách. Spuštění serverové i klientské aplikace na domácím počítači, spuštění serveru na vzdáleném serveru a klienta na domácím počítači nebo spuštění obou aplikací na vzdáleném serveru. Výstupy fungovaly tak jak měly, avšak pro přehlednost a stručnost dokumentace zde uvádím jen pár demonstračních příkladů.
		\subsection{./ipk-server -p 5000 beží na serveru merlin.fit.vutbr.cz}
	\begin{center}
		\includegraphics[scale = 0.8]{priloha1.png}\\
	\end{center}

	\subsection{./ipk-client -h merlin.fit.vutbr.cz -p 5000 -n xkoval14}
	\begin{center}
		\includegraphics[scale = 0.83]{priloha2.png}\\
	\end{center}
	
	\subsection{./ipk-client -h merlin.fit.vutbr.cz -p 5000 -f rysavy}
	\begin{center}
		\includegraphics[scale = 0.83]{priloha3.png}\\
	\end{center}
	
	\subsection{./ipk-client -h merlin.fit.vutbr.cz -p 5000 -l a}
	\begin{center}
		\includegraphics[scale = 0.83]{priloha4.png}\\
	\end{center}
	

	
	\vfill
	%%%%%%%%%%%%%%%%%%%%%%%%%%%%%%%%%%%%%%%%%%%%%%%%%%%%%%%%%%%%%%
	
	\section{Použité zdroje}
	\begin{flushleft}
		[1] Demonstrační soubory, Demo\_C.zip [online]. upraveno: 2018-02-22 [cit. 2018-03-07]. Dostupné z: https://wis.fit.vutbr.cz/FIT/st/course-files-st.php?file=\%2Fcourse\%2FIPK-IT\%2Fother\&cid=11963\par	
	\end{flushleft}	
	\begin{flushleft}
		[2] Počítačové komunikace a sítě, PDF prezentace [online]. [cit. 2018-03-07]. Dostupné z: https://wis.fit.vutbr.cz/FIT/st/course-files-st.php?file=\%2Fcourse\%2FIPK-IT\%2Flectures\%2FIPK2017L-02-APLIKACE.pdf\&cid=11963\par	
	\end{flushleft}
	\begin{flushleft}
		[3] Síťový socket, WIKIPEDIA, [online], upraveno: 14.1.2017 [cit 2018-03-07]. Dostupné z: https://cs.wikipedia.org/wiki/S%C3%AD%C5%A5ov%C3%BD_socket\par	
	\end{flushleft}

\end{document}